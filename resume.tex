\documentclass[a4paper,dvipdfmx]{article}

\usepackage{graphicx}
\usepackage{wrapfig}
\usepackage{url}
\usepackage{cite}
\usepackage{xspace}
\usepackage{booktabs}

\usepackage[margin=2truecm]{geometry}

\usepackage{titlesec}

\titleformat{\section}         % Customise the \section command
  {\Large\scshape\raggedright} % Make the \section headers large (\Large),
                               % small capitals (\scshape) and left aligned (\raggedright)
  {}{0em}                      % Can be used to give a prefix to all sections, like 'Section ...'
  {}                           % Can be used to insert code before the heading
  [\titlerule]                 % Inserts a horizontal line after the heading

\titleformat{\subsection}
  {\large\scshape\raggedright}
  {}{0em}
  {}

\newcommand{\datedsection}[2]{%
  \section[#1]{#1 \hfill #2}%
}
\newcommand{\datedsubsection}[2]{%
  \subsection[#1]{#1 \hfill #2}%
}

\newcounter{pubcount}

\pagestyle{plain}

\begin{document}
\noindent
{\Huge SATO Shigeyuki{\LARGE(佐藤 重幸)}}
\begin{flushleft}
\begin{tabular}[t]{lp{10cm}}
Address (office) & Room 101D1 (10F), Faculty of Engineering Bldg. 2, \newline 7-3-1, Hongo, Bunkyo-ku, Tokyo, Japan (ZIP: 113-8656) \\
Tel. (office) & 03-5841-6762 \\
Email & \url{sato.shigeyuki@mi.u-tokyo.ac.jp}
\end{tabular}
\end{flushleft}

\section{Work Experience}
\datedsubsection{The University of Tokyo}{2018.12--Present}
%
Assistant Professor, Graduate School of Information Science and
Technology (Mathematics and Informatics Center)

\datedsubsection{Kochi University of Technology}{2016.4--2018.11}
%
Research Associate (Postdoc), School of Information

\datedsubsection{The University of Tokyo}{2015.4--2015.3}
%
Project Researcher, Graduate School of Information Science and
Technology

\section{Education}
\datedsubsection{University of Electro-Communications}{2011.4--2015.3}
Ph.D. (Engineering), Department of Communication Engineering and
Informatics, Graduate School of Informatics and Engineering

\datedsubsection{University of Electro-Communications}{2009.4--2011.3}
M.E., Department of Computer Science, Graduate School of Electro-Communications

\datedsubsection{University of Electro-Communications}{2005.4--2009.3}
B.E., Department of Computer Science, School of Electro-Communications

\section{Skills}
\begin{itemize}
 \item Languages: Japanese (native), English (proficient).
 \item Academic expertises in programming languages and programming:
       especially, compilers and parallel programming.
 \item Principal investigation: I have been conducting 4 projects
       granted by JSPS Kakenhi and JST ACT-I as a PI.
 \item Research guidance: I have been co-advising many students of
       different attributes (e..g, bachelor, master, doctor, and foreign
       ones).
 \item Teaching in programming courses: I have been co-organizing a
       large-scale all-campus course on introductory Python programming
       (about 6 to 7 hundred annual participants) and leading
       establishing materials for it; I have solely designed and
       organized another advanced online course on Python programming.
 \item Software development and solution service for teaching: a
       developed online system has been running for one thousand and
       hundreds of student users annually in different courses.
\end{itemize}

\section{Academic Services}
\datedsubsection{IPSJ PRO Steering Committee and Editorial Board for Transactions on Programming}{2023.4--Present}
\subsection{JSSST PPL Workshop}
\begin{flushleft}
\begin{tabular}[t]{lp{12cm}}
Program Committee & 2016, 2019, 2021, and 2022 \\
Organizing Committee & 2017, 2018, and 2021 \\
\end{tabular}
\end{flushleft}
\subsection{ACM SIGPLAN PPoPP 2020 Artifact Evaluation Committee}

\section{Grants}

\datedsubsection{Research on Managed Languages and Runtime Systems to
Utilize Memory with Computational Capabilities}{2022.4--2027.3}
\begin{flushleft}
\begin{tabular}[t]{lp{12cm}}
Grant type & JSPS Kakenhi Kiban (B) \\
Position & Co-Investigator \\
Direct cost & 800,000 JPY (tentative) \\
Reference & \url{https://kaken.nii.ac.jp/en/grant/KAKENHI-PROJECT-22H03566/} \\
\end{tabular}
\end{flushleft}

\datedsubsection{Program synthesis for Processing-in-Memory architectures}{2022.4--2026.3}
\begin{flushleft}
\begin{tabular}[t]{ll}
Grant type & JSPS Kakenhi Wakate \\
Position & Principal investigator \\
Direct cost & 3,500,000 JPY \\
Reference & \url{https://kaken.nii.ac.jp/en/grant/KAKENHI-PROJECT-22K17872/} \\
\end{tabular}
\end{flushleft}

\datedsubsection{Advanced Loop Parallelization and Integrated Vectorization}{2018.4--2022.4}
\begin{flushleft}
\begin{tabular}[t]{ll}
Grant type & JSPS Kakenhi Wakate \\
Position  & Principal investigator \\
Direct cost & 3,200,000 JPY \\
Reference & \url{https://kaken.nii.ac.jp/en/grant/KAKENHI-PROJECT-18K18032//} \\
\end{tabular}
\end{flushleft}

\datedsubsection{自動チューニング可能な一般化N体問題解法枠組みの開発
(Development of an auto-tunable generalized N-body problem solving framework)}{2019.4--2021.3}
\begin{flushleft}
\begin{tabular}[t]{ll}
Grant type & JST ACT-I (Kasoku phase)\\
Position  & Principal investigator \\
Direct cost & 22,000,000 JPY \\
Reference & \url{https://projectdb.jst.go.jp/grant/JST-PROJECT-19189186/} \\
\end{tabular}
\end{flushleft}

\datedsubsection{自動チューニング可能な一般化N体問題解法枠組みの開発
(Development of an auto-tunable generalized N-body problem solving framework)}{2017.10--2019.3}
\begin{flushleft}
\begin{tabular}[t]{ll}
Grant type & JST ACT-I \\
Position  & Principal investigator \\
Direct cost & 3,000,000 JPY \\
Reference & \url{https://projectdb.jst.go.jp/grant/JST-PROJECT-17940532/} \\
\end{tabular}
\end{flushleft}

\newpage
\renewcommand{\refname}{Publications (Refereed)}
\begin{thebibliography}{100}
 \bibitem{jip22:argodsm} Hideshima, T., Sato, S., and Taura, K. 2022. Cost-aware Programming on Page-based Distributed Shared Memory. J. Inf. Process., 30:463–475.
 \bibitem{jip22:compth} Endo, W., Sato, S., and Taura, K. 2022. ComposableThreads: Rethinking User-level Threads with Composability and Parametricity in C++. J. Inf. Process., 30:269–282. (Specially Selected Paper)
 \bibitem{jip21:hyper_gemini} Fujimura, S., Sato, S., and Taura, K. 2021. An Efficient and Scalable Distributed Hypergraph Processing System. J. Inf. Process., 29:812–822.
 \bibitem{pldi21:red_par} Morihata, A. and Sato, S. 2021. Reverse Engineering for Reduction Parallelization via Semiring Polynomials. In Proc. the 42nd ACM SIGPLAN Conference on Programming Language Design and Implementation (PLDI 2021), pp.820–834.
 \bibitem{ipdps21:plex} Li, L., Sato, S., Liu, Q., and Taura, K. 2021. Plex: Scaling Parallel Lexing with Backtrack-Free Prescanning. In Proc. the 35th IEEE International Parallel and Distributed Processing Symposium (IPDPS 2021), pp.693–702.
 \bibitem{ispass21:ib_odp} Fukuoka, T., Sato, S., and Taura, K. 2021. Pitfalls of InfiniBand with On-Demand Paging. In Proc. 2021 IEEE International Symposium on Performance Analysis of Systems and Software (ISPASS 2021), pp.265–275.
 \bibitem{ipdrm20:menps} Endo, W., Sato, S., and Taura, K. 2020. MENPS: A Decentralized Distributed Shared Memory Exploiting RDMA. In Proc. the 4th IEEE/ACM Annual Workshop on Emerging Parallel and Distributed Runtime Systems and Middleware (IPDRM 2020), pp.9–16.
 \bibitem{jip20:centaurus} Sato, S., Ihara, H., and Taura, K. 2020. CENTAURUS: A Dynamic Parser Generator for Parallel Ad Hoc Data Extraction. J. Inf. Process., 28:724–732.
 \bibitem{adbis18:par_xpath} Sato, S., Hao, W., Matsuzaki, K. 2018. Parallelization of XPath Queries using Modern XQuery Processors. In Proc. the 22nd European Conference on Advances in Databases and Information Systems (ADBIS 2018), Short Paper, pp.54–62.
 \bibitem{ngc18:pregel} Sato, S. 2018. On Implementing the Push-Relabel Algorithm on top of Pregel. New Gener. Comput., 36(4):419–449.
 \bibitem{jip17:hadoop} Miyazaki, R., Matsuzaki, K., and Sato, S. 2017. A Generator of Hadoop MapReduce Programs that Manipulate One-dimensional Arrays. J. Inf. Process., 25:841–851.
 \bibitem{aplas16:pyblame} Arai, R., Sato, S., and Iwasaki, H. 2016. A Debugger-Cooperative Higher-Order Contract System in Python. In Proc. the 14th Asian Symposium on Programming Languages and Systems (APLAS 2016), pp.148–168.
 \bibitem{fhpc16:s6raph} Coll Ruiz, O., Matsuzaki, K., and Sato, S. 2016. s6raph: Vertex-Centric Graph Processing Framework with Functional Interface. In Proc. Proceedings of the 5th International Workshop on Functional High-Performance Computing (FHPC 2016), pp.58–64.
 \bibitem{ijpp16:tree_skel} Sato, S. and Matsuzaki, K. 2016. A Generic Implementation of Tree Skeletons. Int. J. Parallel Program., 44(3):686–707.
 \bibitem{icpp15:htm} Kobayashi, T., Sato, S., and Iwasaki, H. 2015. Efficient Use of Hardware Transactional Memory for Parallel Mesh Generation. In Proc. the 44th International Conference on Parallel Processing (ICPP 2015), pp.600–609.
 \bibitem{compsoft15:takano} Takano, Y., Iwasaki, H., and Sato, S. 2015. Design and Implementation of Thunk Recycling in the Glasgow Haskell Compiler. Computer Software, 32(1):253–287, in Japanese.
 \bibitem{gpce14:libdsl} Shioda, M., Iwasaki, H., and Sato, S. 2014. LibDSL: A Library for Developing Embedded Domain Specific Languages in D via Template Metaprogramming. In Proc. the 13th International Conference on Generative Programming: Concepts and Experiences (GPCE 2014), pp.63–72.
 \bibitem{aplas14:sd_dfa} Sato, S. and Morihata, A. 2014. Syntax-Directed Divide-and-Conquer Data-Flow Analysis. In Proc. the 12th Asian Symposium on Programming Languages and Systems (APLAS 2014), pp.392–407.
 \bibitem{pro13:tree_par} Sato, S. and Matsuzaki, K. 2013. An Operator Generator for Skeletal Programming on Trees. IPSJ Trans. PRO, 6(4):38–49, in Japanese.
 \bibitem{pldi11:red_par} Sato, S. and Iwasaki, H. 2011. Automatic Parallelization via Matrix Multiplication. In Proc. the 32nd ACM SIGPLAN conference on Programming Language Design and Implementation (PLDI 2011), pp.470–479.
 \bibitem{aplas09:gpu_skel} Sato, S. and Iwasaki, H. 2009. A Skeletal Parallel Framework with Fusion Optimizer for GPGPU Programming. In Proc. the 7th Asian Symposium on Programming Languages and Systems (APLAS 2009), pp.79–94.
 \setcounter{pubcount}{\theenumiv}
\end{thebibliography}

\renewcommand{\refname}{Publications (Posters, Non-refereed)}
\begin{thebibliography}{100}
 \setcounter{enumiv}{\thepubcount}
 \bibitem{splash22:mvnb} Nakamaru, T. and Sato, S. 2022. Multiverse Notebook: A Notebook Environment for Safe and Efficient Exploration. In Proc. the 2022 ACM SIGPLAN International Conference on Systems, Programming, Languages, and Applications: Software for Humanity (SPLASH Companion 2022), pp.7–8. (Extended Poster Abstract)
 \bibitem{ispass22:vipp} Sato, S., Iizuka, K., Yoshifuji, N., and Natsume, M. 2022. VIPP: Validation-Included Precision-Parametric N-Body Benchmark Suite. In Proc. 2022 IEEE International Symposium on Performance Analysis of Systems and Software (ISPASS 2022), pp.156–158. (Extended Poster Abstract)
 \bibitem{cloudcomp20:xpath} Hao, W., Matsuzaki, K., and Sato, S. 2021. A Dual-Index Based Representation for Processing XPath Queries on Very Large XML Documents. In Proc. the 10th EAI International Conference on Cloud Computing (CloudComp 2020), pp.18–30.
 \bibitem{splash19:nbody} Sato, S. 2019. A Symmetry-Based N-Body Solver Compiler. In Proc. the 2019 ACM SIGPLAN International Conference on Systems, Programming, Languages, and Applications: Software for Humanity (SPLASH Companion 2019), pp.21–22. (Extended Poster Abstract)
\end{thebibliography}
\end{document}

